

%This document provides an introduction to statistical disclosure control (SDC)
%and guidelines on how to apply SDC methods to microdata. Section 1 introduces
%basic concepts and presents a general workflow. Section 2 discusses methods of
%measuring disclosure risks for a given micro dataset and disclosure scenario. Sec-
%tion 3 presents some common anonymization methods. Section -1 introduces how
%to assess utility of a micro dataset after applying disclosure limitation methods.
\documentclass{beamer}

\usepackage{amsmath}
\usepackage{amssymb}

\begin{document}
\begin{frame}

\frametitle{Microdata}
\textbf{MicroData}
\begin{itemize}
	\item
A microdata file is a dataset that holds information collected on individual units;
examples of units include people, households or enterprises. 
\item For each unit, a set of
variables is recorded and available in the dataset. 
%\item This section discusses concepts
%related to disclosure and SDC methods, and provides a workfiow that shows how
%to apply SDC methods to microdata.
\end{itemize}
\end{frame}
%================================================================== %
\subsection*{1.1. Categorization of Variables}
\begin{frame}
\frametitle{Types of variables}

In accordance with disclosure risks, variables can be classified into three groups,
which are not necessarily exclusive:

\bigskip
\begin{description}
	\item[Direct Identifiers] are variables that precisely identify statistical units. 
\end{description}

\begin{itemize}
	\item For example, social insurance numbers, names of companies or persons and addresses
	are direct identifiers 
	\item (Remark: Primary Keys)
\end{itemize}
\end{frame}
%======================================================================== %
\begin{frame}
\frametitle{Types of variables}
\begin{description}
	\item[Key variables] are a set of variables that, when considered together, can be used
	to identify individual units. For example, it may be possible to identify
	individuals by using a combination of variables such as gender, age, region
	and occupation. Other examples of key variables are income, health status,
	nationality or political preferences. 
\end{description}
\end{frame}
%======================================================================== %
\begin{frame}
	\frametitle{Types of variables}
\begin{itemize}
	\item Key variables are also called implicit
	identifiers or quasi-identifiers. \item When discussing SDC methods, it is preferable
	to distinguish between categorical and continuous key variables based on the
	scale of the corresponding variables.
	\item Non-identifying variables are variables that are not direct identifiers or key variables.
\end{itemize}
%======================================================= %
%For specific methods such as l-diversity, another group of sensitive variables is defined in Section 
\end{frame}
\subsection*{1.2. What is disclosure?}
\begin{frame}
\frametitle{Types of Disclosure}

\begin{itemize}
\item In general, disclosure occurs when an intruder uses the released data to reveal
previously unknown information about a respondent.
\item There are three different types of disclosure:\textit{(next set of slides)}
\end{itemize}


\end{frame}
%==================================================================== %
\begin{frame}
	\frametitle{Types of Disclosure}

\textbf{Identity disclosure:}
\begin{itemize}
	\item In this case, the intruder associates an individual with a re-
	leased data record that contains sensitive information, i.e. linkage with external available data is possible.
	\item Identity disclosure is possible through direct
	identfiers, rare combinations of values in the key variables and exact knowledge of continuous key variable values in external databases. 
	\item For the latter,
	extreme data values 
	lead to high re-identification risks, i.e. it is likely that responends with extreme data values are disclosed.
\end{itemize}
% - (e.g., extremely high turnover values for an enterprise)
\end{frame}
%==================================================================== %
\begin{frame}
	\frametitle{Types of Disclosure}
\textbf{Attribute disclosure:}
\begin{itemize}
	
	\item In this case, the intruder is able to determine some characteristics of an individual based on information available in the released data.
	\item For example, if all people aged 56 to GO who identify their race as black in
	region 12345 are unemployed, the intruder can determine the value of the
	variable labor status.
	\item \textbf{Also} Diet - detailed questionnaire on diet can give clues to other aspects of some people's lives.
\end{itemize}
\end{frame}
%==================================================================== %
\begin{frame}
	\frametitle{Types of Disclosure}
\textbf{Inferential disclosure:}
\begin{itemize}
	
	\item In this case, the intruder, though with some uncertainty,
	can predict the value of some characteristics of an individual more accurately with the released data.
\end{itemize}
\end{frame}
%==================================================================== %
\begin{frame}
\frametitle{Linkage}
\textbf{Linkage}
\begin{itemize}
	
\item If linkage is successful based on a number of identifiers, the intruder will have
	access to all of the information related to a specific corresponding unit in the
	released data. 
\item This means that a subset of critical variables can be exploited to
	disclose everything about a unit in the dataset.
\end{itemize}

\end{frame}
%==================================================================== %
\end{document}