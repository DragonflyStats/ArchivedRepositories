\begin{itemize}
\item l-diversity is a form of group based anonymization that is used to preserve privacy in data sets by reducing the granularity of a data representation.
\item This reduction is a trade off that results in some loss of effectiveness of data management or mining algorithms in order to gain some privacy. The l-diversity model is an extension of the k-anonymity model which reduces the granularity of data representation using techniques including generalization and suppression such that any given record maps onto at least k other records in the data. 
\item The l-diversity model handles some of the weaknesses in the k-anonymity model where protected identities to the level of k-individuals is not equivalent to protecting the corresponding sensitive values that were generalized or suppressed, especially when the sensitive values within a group exhibit homogeneity. 
\item The l-diversity model adds the promotion of intra-group diversity for sensitive values in the anonymization mechanism.
\end{itemize}
%============================================%
\end{document}