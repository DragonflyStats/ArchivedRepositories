\documentclass[SDCmaster.tex]{subfiles}
\begin{document}
	\begin{frame}
		\frametitle{sdcMicro}
\begin{itemize}
\item The \texttt{R} package \textbf{sdcMicro} is a well-known collection of microdata protection methods developed by Statistics Austria, which is already in use in several national statistics offices. 
\item sdcMicro has become one of the standard tools for microdata protection during the last five years.
\item The IHSN is supporting the further development of sdcMicro and has partnered with its developers to perform the following tasks (next slide).
\end{itemize}
\end{frame}
%=====================================%

\begin{frame}
	\begin{itemize}
\item Include in \textbf{sdcMicro} relevant methods available in the IHSN plug-ins
\item Test \textbf{sdcMicro} on real datasets to calibrate its outputs and facilitate their interpretation
\item Develop practical guidelines to support the use of a toolbox and help users navigate between methods and associated algorithms

	\end{itemize}
SdcMicro already includes several popular methods for microdata anonymization; some of these methods can also be found in the IHSN C++ plug-ins.
\end{frame}
%=====================================%

\begin{frame} 
\begin{itemize}
\item The overlapping methods have been tested and compared with their analogous implementation in \textbf{sdcMicro}. 
\item Three new methods (or improved implementations) have been included in sdcMicro: suda2 (i.e., finding minimal samples unique), rank swap (i.e., numerical rank swapping), and mdav (i.e., micro-aggregation). 
\item Since the C++ code contained specific class structures 
and required multiple and sometimes different header files to be included when compiling the code, the inclusion of these new methods into \texttt{R} has proved to be a complex task. 
\end{itemize}
\end{frame}
%=====================================%

\begin{frame}The following Figure 1 shows computation time efficiency gains between the old and new implementation of rank swapping in sdcMicro, based on 100 runs on a 10-dimensional dataset with varying numbers of observations.

Figure 1 computation time efficiency gains between old and new implementation of rank swapping algorithms

SDC
\end{frame}
%=====================================%

\begin{frame}
	\begin{itemize}
\item Version 4.4.0 of the sdcMicro package is available on the Comprehensive R Archive Network (CRAN).
\item Existing guidelines and a user guide for \textbf{sdcMicro} are being updated. 
\item A specific tutorial is being developed to show how to implement these concepts and algorithms on real datasets. 
	\end{itemize}


\end{frame}
%=====================================%

\begin{frame}
	\frametitle{Statistical Disclosure Control}
	\begin{itemize}
\item This tutorial is being drafted with examples of the European Union SES dataset.
\item The IHSN is promoting the adoption of sdcMicro and an associated guidelines toolbox for the creation of Public Use Files and Scientific Use Files.
\item See our pages Software - Statistical Disclosure Control and Guidelines on Microdata Anonymization.
	\end{itemize}

\end{frame}
%=====================================%
\begin{frame}
	\begin{itemize}
\item Class \texttt{"sdcMicroObj"}
\item Data shuffling and General Additive Data Perturbation.
\item \texttt{suda2} Detecting Special Uniques
	\end{itemize}


\end{frame}
%=====================================%
\begin{frame}[fragile]
\begin{verbatim}
> data(testdata)
> sdc <- createSdcObj(testdata,
 + keyVars=c('urbrur','roof','walls','relat','sex'),
 + pramVars=c('water','electcon'),
 + numVars=c('expend','income','savings'), w='sampling_weight')
> fk=freq(sdc)
> Fk=freq(sdc,type="Fk")
\end{verbatim}
\end{frame}
%=====================================%
\begin{frame}[fragile]
\begin{verbatim}
> print(sdc)
Number of observations violating

-  2-anonymity:  26 
-  3-anonymity:  52 
--------------------------

Percentage of observations violating
-  2-anonymity:  0.57 % 
-  3-anonymity:  1.14 % 
\end{verbatim}
\end{frame}
%=====================================%
\begin{frame}[fragile]
\begin{verbatim}
> print(sdc,type="ls")

urbrur .. 0 [ 0 %]

roof .... 0 [ 0 %]

walls ... 0 [ 0 %]

relat ... 0 [ 0 %]

sex ..... 0 [ 0 %]
\end{verbatim}
\end{frame}
%=====================================%
\begin{frame}[fragile]
	\begin{verbatim}
> print(sdc,type="recode")
Reported is the
number | mean size and | size of smallest category

-------------
urbrur .. 2 | 2290 | 646 
-------------
roof .... 5 | 916 | 16 
-------------
walls ... 3 | 1527 | 50 
-------------
relat ... 9 | 509 | 1 
-------------
sex ..... 2 | 2290 | 2284 
\end{verbatim}
\end{frame}
%=====================================%
\begin{frame}[fragile]
	\begin{verbatim}
> print(sdc,type="risk")

--------------------------
0 obs. with higher risk than the main part
Expected no. of re-identifications:
2.41 [ 0.05 %]
--------------------------

\end{verbatim}
\end{frame}
%=====================================%
\begin{frame}[fragile]
	\begin{verbatim}
> print(sdc,type="numrisk")
Disclosure Risk is between: 
[0% ; 100%] (current)
- Information Loss:
IL1: 0
- Difference Eigenvalues: 0 %
\end{verbatim}
\end{frame}
%=====================================%
\begin{frame}[fragile]
	\begin{verbatim}
> print(sdc,type="pram")
PRAM has not been applied!
> 
\end{verbatim}
\end{frame}
\end{document}