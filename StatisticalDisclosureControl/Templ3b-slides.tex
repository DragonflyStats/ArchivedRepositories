\documentclass{beamer}

\usepackage{amsmath}
\usepackage{graphicx}
\usepackage{amssymb}

\begin{document}
\begin{frame}
\frametitle{5. Adding Noise}
\begin{itemize}

\item Adding noise is a perturbative protection method for microdata, which is typically
applied to continuous variables. 
\item \textbf{Motivation:} This approach protects data against exact matching with external files if, for example, information on specific variables is available
from registers.

\item While this approach sounds simple in principle, many different algorithms can
be used to overlay data with stochastic noise. 
\item It is possible to add uncorrelated
random noise. In this case, the noise is usually distributed and the variance of
the noise term is proportional to the variance of the original data vector. 
\end{itemize}
\end{frame}
%================================================ %
\begin{frame}
	\frametitle{5. Adding Noise}
	\begin{itemize}
		\item \textbf{Important} - Adding
uncorrelated noise preserves means, but variances and correlation coefficients between variables are not preserved.
\item This statistical property is respected, however,
if correlated noise method(s) are applied.

\item For the correlated noise method , the noise term is derived from a
distribution having a covariance matrix that is proportional to the co-variance matrix of the original microdata. 
\item In the case of correlated noise addition, correlation
coefficients are preserved and at least the co-variance matrix can be consistently
estimated from the perturbed data. 
\end{itemize}
\end{frame}
%================================================ %
\begin{frame}
	\frametitle{5. Adding Noise}
	\textbf{Implementation}
	\begin{itemize}
\item The data structure may differ a great deal,
however, if the assumption of normality is violated. 
\item Since this is virtually always
the case when working with real-world datasets, a robust version of the correlated
noise method is included in \textbf{sdcMicro}. 
% \item This method allows departures from model
% assumptions and is described in detail in Templ and Meindl [2008b]). 
\item Noise can be added to variables in \textbf{sdcMicro} using function \texttt{addNoise()} or by
using sdcMicroGUI.
\item More information can be found in the help file by calling \texttt{?addNoise} or using the graphical
user interface help menu.
\end{itemize}
\end{frame}
%================================================ %
\begin{frame}
	\frametitle{5. Adding Noise}
	\textbf{Outliers / Implementation}
	\begin{itemize}
\item In \textbf{sdcMicro}, several other algorithms are implemented that can be used to add
noise to continuous variables. For example, it is possible to add noise only to outlying observations. 
\item In this case, it is assumed that such observations possess higher risks than non-outlying observations.
%% Page 17/ 31

 
\item Other methods ensure that the amount of
noise added takes into account the underlying sample size and sampling weights.

\end{itemize}
\end{frame}
%================================================ %
\begin{frame}
	\frametitle{6. Shuffling}
	\begin{itemize}
\item \textbf{Mathematically Complex}
Various masking techniques based on linear models have been developed in literature, such as multiple imputation, general additive data perturbation and the information preserving statistical obfuscation syrithetic data generators.
\item These methods are capable of maintaining
linear relationships between variables but fail to maintain marginal distributions
or non-linear relationships between variables.

\item Several methods are available for shuffling in \textbf{sdcMicro} and \textbf{sdcMicroGUI}, whereas
the first (default) one (ds) is recommended to use. 
%The explanation of all these methods goes far beyond this guidelines and interested readers might read the
% original paper from Muralidliar and Sarathy [Z006]. In the following only a brief
% introduction to shuffling is given.

\end{itemize}
\end{frame}
%================================================ %
\begin{frame}
	\frametitle{6. Shuffling}
	\begin{itemize}
		\item 
\item Shuffling simulates a synthetic value of the continuous key variables conditioned on independent, non-confidential variables. 
\item After
the simulation of the new values for the continuous key variables, reverse mapping
(shuffling) is applied. 
\item This means that ranked values of the simulated values are
replaced by the ranked values of the original data (columnwise).

\end{itemize}
\end{frame}

%================================================ %
\begin{frame}
	\frametitle{6. Shuffling}
	\begin{itemize}
\item \textbf{Example} To explain this theoretical concept more practically we can assume that we have
two continuous variables containing sensitive information on income and savings.
\item These variables are used as regressors in a regression model where suitable variables
are taken as predictors, like age, occupation, race, education. 
\item Of course it is of
crucial to find a good model having good predictive power. 
\end{itemize}
\end{frame}
%================================================ %
\begin{frame}
	\frametitle{6. Shuffling}
	\begin{itemize}
\item New values for the
continuous key variables, income and savings, are simulated based on this model.

%[for details, have a look at Xluralidliar and Sarathy, 2006]. 
\item However, these expected
values are not used to replace the original values. 
\item Rather a value re-assignment system (shuffling) of the original
values using the generated values is carried out. 


\end{itemize}
\end{frame}
%================================================ %
\begin{frame}
	\frametitle{6. Shuffling}
\textbf{Value Assignment System}
	\begin{itemize}
		\item This approach (reverse mapping)
is applied to each sensitive variable can be summarized in the following steps:
\begin{enumerate}
\item rank original variable
\item rank generated variable
\item for all observations, replace the fitted value of the variable with rank i
with the value of the original sensitive variable with rank $i$.
\item once finished, the modified variable contains only original values and is finally
used to replace the original sensitive variable.
\end{enumerate}
\end{itemize}
\end{frame}

%================================================ %
\begin{frame}
	\frametitle{6. Shuffling}
	\begin{itemize}
\item It can be shown that the structure of the data is preserved when the model fir
is of good quality. \item In the implementation of sdcMicro, a model of almost any form
and complexity can be specified 
\\ (see \texttt{?shuffling} for details).
\end{itemize}
\end{frame}
\end{document}