The Special Uniques Methodology
The concept of the “special unique” was coined by Elliot et al (1998). The behind the
concept principle is that a microdata record which is sample unique on courser, less
detailed information is more risky than one which is unique on a finer, more detailed
information. A particular case of that is where a record which is sample unique on a
set of variables K and is also unique on a subset of K. Such a record is called a
special unique, with respect of variable set K.
Extensive empirical work (Elliot 2000, Elliot and Manning 2001, Merrett et al 2005)
has shown that special uniques are more likely to be population unique than random
uniques. Further work (e.g. Elliot et al 2002) has shown that it is possible to classify
special uniques according to the size and number of subsets which are unique
minimal sample uniques (MSU) and that such classifications are correlated with the
reciprocal population equivalence class, which is a generally accepted measure of
underlying risk. 
