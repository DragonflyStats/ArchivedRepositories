


%4 MEASURING DATA UTILITY AND INFORMATION LOSS
\documentclass[]{article}

%opening
\title{}
\author{}

\begin{document}
%==========================================================%
\section*{4. Measuring Data Utility and Information Loss}
Measuring data utility of the microdata set after disclosure limitation methods
have been applied is encouraged to assess the impact of these methods.

%% Page 18 / 31
%% 4 MEASURING DATA UTILITY AND INFORMATION LOSS
\subsection*{4.1. General applicable methods}
Anonymized data should have almost the same structure of the original data and
should allow any analysis with high precision.

To evaluate the precision, use various classical estimates such as means and co-
variances. Using function \texttt{dUtility()}, it is possible to calculate different measures
based on classical or robust distances for continuous scaled variables. Estimates are
computed for both the original and perturbed data and then compared. Following
are three important information loss measures:

%------------------------------------------------------------------------------------%
\begin{itemize}
\item \textbf{IL1s} is a measures that can be interpreted as scaled distances
between original and perturbed values for all p continuous key variables.

\item \textbf{eig} is a measure calculating relative absolute differences between eigenvalues
of the co-variances from standardized continuous key variables of the original
and perturbed variables. Eigenvalues can be estimated from a robust or
classical version of the co-variance matrix.
\item lm is a measure based on regression models, similar to a residual.
% It is defined as  — /with ii,” being fitted values from a pre-specified model obtained from the
% original (index 0) and the modified data (index m). Index w indicates that
% the survey weights should be considered When fitting the model.
\end{itemize}
Note that these measures are automatically estimated in sdcMicro when an
object of class sdcMicr0Obj is generated or whenever continuous key variables are
modified in such an object. Thus. no user input is required. We note however
that only the former two measures are automatically presented in the GUI in tab
Continuous) as IL1 and Difference Eigenvalues respectively.
\newpage
%=========================================%
\subsection*{4.2. Specific tools}
In practice, it is not possible to create an anonymized file with the same structure as
the original file. An important goal, however, should always be that the difference
in results of the most important statistics based on anonymized and original data
should be very small or even zero. Thus, the goal is to measure the data utility
based on benchmarking indicators, which
is in general a better approach to assess data quality than applying general tools.
The first step in quality assessment is to evaluate what users of the underlying
data are analyzing and then try to determine the most important estimates, or
benchmarking indicators.
% [see, e.g., Templ, 2011b,-a]. 

Special emphasis should be
put on benchmarking indicators that take into account the most important vari-
ables of the micro dataset. Indicators that refer to the most sensitive variables
within the microdata should also be calculated. 


The general procedure is quite
simple and can be described in the following steps:
\begin{enumerate}
\item Selection of a set of (benchmarking) indicators
\item Choice of a set of criteria as to how to compare the indicators
\item Calculation of all benchmarking indicators of the original micro data
\item Calculation of the benchmarking indicators on the protected micro data set
\item Comparison of statistical properties such as point estimates, variances or
overlaps in confidence intervals for each benchmarking indicator
\item Assessment as to whether the data utility of the protected micro dataset is
good enough to be used by researchers
\end{enumerate}
%%Page 19 / 31
%% 4 MEASURING DATA UTILITY AND INFORMATION LOSS
\begin{itemize}

%-------------------------------------------------------------------------%
\item If the quality assessment in the last step of the sketched algorithm is satisfactory,
the anonymized micro dataset is ready to be published. If the deviations of the
main indicators calculated from the original and the protected data are too large,
the anonymization procedure should be restarted and modified. It is possible to
either change some parameters of the applied procedures or start from scratch and
completely change the anonymization process.
%-----------------------------------------------------------%
\item Usually the evaluation is focused on the properties of numeric variables, given
unmodified and modified microdata. It is of course also possible to review the impact of local suppression or recoding that has been conducted to reduce individual re-identification risks. 
\item Another possibility to evaluate the data utility of numerical
variables is to define a model that is fitted on the original, unmodified microdata.
\item The idea is to predict important, sensitive variables using this model both for the
original and protected micro dataset as a first step. 


%-----------------------------------------------------------%
\item In a second step, statistical
properties of the model results, such as the differences in point estimates or vari-
ances, are compared for the predictions, given original and modified microdata,
thcn the resulting quality is assessed. If the deviations arc small enough, one may
go on to publish the safe and protected micro dataset. Otherwise, adjustments
must be made in the protection procedure. This idea is similar to the information
loss measure lm described in Section 4.1.
%-------------------------------------------------------------------%
\item In addition, it is interesting to evaluate the set of benchmarking indicators not
only for the entire dataset but also independently for subsets of the data. In
this case, the microdata are partitioned into a set of h groups. 
\item The evaluation of
benchmarking indicators is then performed for each of the groups and the results
are evaluated by reviewing differences between indicators for original and modified
data in each group. 
%Templ ct al. [20l~1a] gives a detailed description of ber1cl1marking indicators for the SES data. An excerpt of this study is shown in the appendix.
\end{itemize}

%--------------------------------------------------------------------%
\newpage
\subsection*{4.3. Workflow}
Figure 3 outlines the most common tasks, practices and steps required to obtain
confidential data. The steps are summarized here:
\begin{enumerate}
\item The first step is actually to make an inventory of other datasets available to
users, to decide on what an acceptable level of risk will be, and to identify
the key users of the anonymized data to make decisions on anonymisation to
achieve high precision on their estimates on the anonymized data.
\item The first step in anonyimization is always to remove all direct identification
variables and variables that contain direct information about units from the
microdata set.
\item  Second, determine the key variables to use for all risk calculations. This
decision is subjective and often involves discussions with subject matter spe-
cialists and interpretation of related national laws. 

%
%Please see Tcmpl ct al.
%[201-la] for practical applications on how to define key variables. Note that

%% Page 20 / 31
%% 4 MEASURING DATA UTILITY AND INFORMATION LOSS


%Possibilites for anonyinising micro data using different SDC methods

\item The most important methods are included in the sdcMicroGUI, such
as basic risk measurement, recoding, local suppression, PRAM (post-
randomization), information loss measures, shuffling, microaggregation
and adding noise. Other methods listed in the figure for the sake of
completeness are included in the sdcMicro R package and in the sim-
Population R package.
for the simulation of fully synthetic data, choosing key variables is not nec-
essary since all variables are produced synthetically, see for example Alfons
ct al. [2011].
\item After the selection of key variables, measure disclosure risks of individual
units. This includes the analysis of sample frequency counts as Well as the
application of probability methods to estimate corresponding individual re-
identification risks by taking population frequencies into account.
\item Next, modify observations with high individual risks. Techniques such as
recoding and local suppression, recoding and swapping, or PRAM can be
applied to categorical key variables. In principle, PRAM or swapping can
also be applied without prior recoding of key variables; a lower swapping rate
might be possible, however, if recoding is applied before. The decision as to
which method to apply also depends on the structure of the key variables. In
general, one can use recoding together with local suppression if the amount
of unique combinations of key variables is low. PRAM should be used if
the number of key variables is large and the number of unique combinations
is high; for details, see Sections 3.1 and 3.3 and for practical applications
Tcmpl ct al. [2014a]. The values of continuously scaled key variables must be
perturbed as well. In this case, micro-aggregation is always a good choice (sec
Section  More sophisticated methods such as shuffling (see Section 3.6
often provide promising results but are more complicated to apply.
\item  After modifying categorical and numerical key variables of the microdata,
estimate information loss and disclosure risk measures. The goal is to re-
lease a safe micro dataset with low risk of linking confidential information
to individuals and high data utility. If the risks is below a tolerable risk and

%% Page 21 / 31
%% 4 MEASURING DATA UTILITY AND INFORMATION LOSS
the data utility is high, the anonymized dataset is ready for release. Note
that the tolerable risk depends on various factors like national laws and sen-
sitivity of data, but also subjective arbitrary factors play a role and the risk
depends on the selected key variables — the disclosure scenario. If the risk
is too high or the data utility is too low, the entire anonymization process
must be repeated, either with additional perturbations if the remaining re-
identification risks are too high, or with actions that will increase the data
utility.
\end{enumerate}
\newpage
In general, the following recommendations hold:
\begin{description}
\item[Recommendation 1:] Carefully choose the set of key variables using knowledge
of both subject matter experts and disclosure control experts. As already men-
tioned, the key variables are those variables for which an intruder may possible
have data/ information, e.g. age and region from persons or turnover of enter-
prises. Which external data are available containing information on key variables
is usually known by subject matter specialist.
\item[Recommendation 2:] Always perform a frequency and risk estimation to evaluate
how many observations have a high risk of disclosure given the selection of key
variables.
\item[Recommendation 3:] Apply recoding to reduce uniqueness given the set of cat-
egorical key variables. This approach should be done in an exploratory manner.
Receding on a variable, however, should also be based on expert knowledge to
combine appropriate categories. Alternatively, swapping procedures may be ap-
plied on categorical key variables so that data intruders cannot be certain if an
observation has or has not been perturbed.
\item[Recommendation 4:] If recoding is applied, apply local suppression to achieve
k-anonymity. In practice, parameter It" is often set to 3.
\item[Recommendation 5:] Apply micro-aggregation to continuously scaled key vari-
ables. This automatically provides k-anonymity for these variables.
\item[Recommendation 6:] Quantify the data utility not only by using typical esti-
mates such as quantiles or correlations, but also by using the most important
data-specific benchmarking indicators
\end{description}
\end{document}
 (see Section 
Recoding and micro-aggregation work well to obtain non-confidential data with
high data quality. While the disclosure risks cannot be calculated in a meaningful
way if probabilistic methods (e.g. PRAM) have been applied, these methods are
advantageous whenever a large number of key variables is selected. This is because
a high number of key variables leads to a high number of unique combinations
that cannot be significantly reduced by applying recoding. More on assessing data
quality can be found in section 4.2.
%% Page 22 / 31
