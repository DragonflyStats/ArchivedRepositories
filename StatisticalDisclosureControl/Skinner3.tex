

Expressions for the risk measures in (8) and (9) in terms of ,5 are provided by Skinner
and Holmes (1998) and Elamir and Skinner (2006). Assumptions about the sampling
scheme are required to estimate /1’. Under Bernoulli sampling with inclusion probability
71', it follows from (6) that fk ll, ~ Po(7r2., ). Assuming also (10), fl may be estimated
by standard maximum likelihood methods. 

A simple extension of this argument also
applies under Poisson sampling where the inclusion probability rrk may vary with respect
to the key variables, for example if a stratifying variable is included among the key
variables. In this case, we have fk I), ~ Po(/112.‘). Skinner and Shlomo (2008) discuss
methods for the specification of the model in (10). Skinner (2007) discusses the possible
dependence of the measure on the search method employed by the intmder.
%===========================================%
\subsection*{3.4. SDC methods}
In this section we summarize a number of SDC methods for survey microdata.
%===========================================%
\noindent \textbf{Transformation of variables to reduce detail}\\ Categorical key variables may be
transformed, in particular, by combining categories. For example, the variable household
size might be top coded by creating a single maximum category, such as 8+. Continuous
key variables may be banded to form ordinal categorical variables by specifying a series
of cut-points between which the intervals define categories. 
The protection provided by combining categories of key variables can be assessed following the methods in sections
3.2. and 3.3. See also Reiter (2005). Provided the transformation is clear and explicit,
this SDC method has the advantage that the reduction of utility is clear to the data user,
who may suffer loss of information but the validity of analyses is not damaged.
%===========================================%
\noindent \textbf{Stochastic perturbation of variables}\\ The values of potential key variables are
perturbed in a stochastic way. In the case of continuous variables perturbation might
involve the addition of noise, analogous to the addition of measurement error (Sullivan and Fuller, 1989; Fuller, 1993). In the case of categorical variables, perturbation may consist of misclassification, termed the \textbf{Post Randornisation Method (PRAM)} by
Gouweleeuw et al. (1998). 

Perturbation may be undertaken in a way to preserve specified features of the microdata, e.g. the means and standard deviations of variables in the perturbed microdata may be the same as in the original microdata, but in practice there
will inevitably be unspecified features of the microdata which are not reproduced. For example, the estimated correlation between a perturbed variable and an unperturbed variable will often be downwardly biased if an analyst uses the perturbed data but ignores
the fact that perturbation has taken place. 

%===========================================%
An altemative is to provide users with the
precise details of the perturbation method, including parameter values such as the
standard deviation of the noise or the entries in the misclassification matrix, so that they
may ‘undo’ the impact of perturbation when undertaking their analyses. See e.g. Van den
Hout and Van der Heijden (2002) in the case of PRAM or Fuller (1993) in the case of
added noise. In principle, this may permit valid analyses although there will usually be a
loss of precision and the practical disadvantages are significant.
%===========================================%
Synthetic microdata: This approach is similar to the previous one, except that the aim
is to avoid requiring special methods of analysis. Instead, the values of variables in the
file are replaced by values generated from a model in a way that is designed for the
analysis of the synthetic data, as if it were the true data, to generate consistent point
estimates (under the assumption that the model is valid). The model is obtained from
fitting to the original microdata. In order to enable valid standard errors as well as
consistent point estimators, Raghunathan et al. (2003) propose that multiple copies of the
synthetic microdata are generated in such a way that multiple imputation methodology
can be used. See Reiter (2002) for discussion of complex designs. Abowd and Lane
(2004) discuss release strategies combining remote access to one or more such synthetic
microdata files with much more restricted access to the original microdata in a safe
setting.
%===========================================%
Selective perturbation. Often concern focuses only on records deemed to be risky and
it may be expected that utility will be greater if only a subset of risk records is perturbed.
In addition to creating stochastically perturbed or synthetic values for only targeted
records, it is also possible just to create missing values in these records, called local
ruppresrion by Willenborg and de Waal (2001), or both to create missing values and to
replace these by imputed values, called blank and impure by Federal Committee on
Statistical Methodology (2005). A major problem with such methods is that they are
likely to create biases if the targetted values are unusual. The data user will typically not
be able to quantify these biases, especially when the records selected for blanking depend
on the values of the variable(s) which are to made missing. Reiter (2003) discusses how
valid inference may be conducted if multiple imputed values are generated in a specified
way for the selected records. He refers to the resulting data as partially synthetic
microdata.
%===========================================%
\noindent \textbf{Record Swapping}\\
The previous methods focus on the perturbation of the values of
the variables for all or a subset of records. The method of record swapping involves,
instead, the values of one or more key variables being swapped between records. The
choice of records between which values are swapped may be controlled so that certain
bivariate or multivariate frequencies are maintained (Dalenius and Reiss, 1982) in
particular by only swapping records sharing certain characteristics (Willenborg and de
Waal, 2001, sect. 5.6). In general, however, it will not be possible to control all
multivariate relationships and record swapping may damage utility in an analogous way
to misclassification (Skinner and Shlomo, 2007). Reiter (2005) discusses the impact of
swapping on identification risk.
%===========================================%
Microaggregation. This method (Defays and Anwar, 1998) is relevant for continuous
variables, such as in business survey microdata, and in its basic form consists of ordering
the values of each variable and forming groups of a specified size k (the first group
contains the $k$ smallest values, the second group the next k smallest values and so on).

The method replaces the values by their group means, separately for each variable. An
advantage of the method is that the modification to the data will usually be greatest for
outlying values, which might also be deemed the most risky. It is difficult, however, for
the user to assess the biasing impact of the method on analyses.

SDC methods will generally be applied after the editing phase of the survey, during
which data may be modified to meet certain edit constraints. The application of some
SDC methods may, however, lead to failure of some of these constraints. Shlomo and de
Waal (2006) discuss how SDC methods may be adapted to take account of editing
considerations.

%===========================================%
\newpage
\section{3.5 SDC for survev weights and other design information}
Survey weights and other complex design information are often released with survey
microdata in order that valid analyses can be undertaken. It is possible, however, that
such design information may contribute to disclosure risk. For example, suppose a survey
is stratified by a categorical variable X with different sampling fractions in different
categories of X. Then, if the nature of the sampling design is published (as is common), it
may be possible for the intruder to determine the categories of X from the survey weight.
Thus, the survey design variable may effectively become a key variable. See de Waal and
Willenborg (1997) and Willenborg and de Waal (2001, sect. 5.7) for further discussion of
how survey weights may lead to design variables becoming key variables. 

Note that this does not imply that survey weights should not be released; it just means that disclosure
risk assessments should take account of what information survey weights may convey.
Willenborg and de Waal (2001, sect. 5.7.3) and Mitra and Reiter (2006) propose some approaches to adjusting weights to reduce risk.

In addition to the release of survey weights, it is common to release either stratum or
primary sampling unit (PSU) labels or replicate labels, to enable variances to be estimated. These labels will generally be arbitrary and will not, in themselves, convey any identifying information. Nevertheless, as for survey weights, the possibility that they
could be used to convey information indirectly needs to be considered. For example, if the PSUs are defined by areas for which public information is available, e.g. a property tax rate, and the microdata file includes area-level variables, then it is possible that these variables may enable a PSU to be linked to a known area. As another example, suppose that a PSU is an institution, such as a school, then school level variables on the microdata file, such as the school enrolment size, might enable the PSU to be linked to a known institution. Even for individual level microdata variables, it is possible that sample-based
estimates of the total or mean of such variables for a stratum, say, could be matched to
published values, allowing for sampling uncertainty.
%===========================================%
A standard simple approach to avoiding releasing PSU or replicate identifiers is to
provide information on design effects or generalized variance functions instead. Such
methods are often inadequate, however, for the full range of uses of survey microdata
(Yung, 1997). Some possible more sophisticate approaches include the use of adjusted
bootstrap replicate weights (Yung, 1997), adjusted pseudo-replicates or pseudo PSU
identifiers (Dohrmann et al., 2002), or combined stratum variance estimators (Lu et al.,
2006).
%===========================================%
\subsection{4. Conclusion}
The development of SDC methodology continues to be stimulated by a wide range of
practical challenges and by ongoing innovations in the ways that survey data are used,
with no signs of diminishing concerns about confidentiality. There has been a tendency
for some SDC methods to be developed in somewhat ad hoc way to address specific
problems and one aim of this paper has been to draw out some principles and general
approaches which can guide a more unified methodological development. 

Statistical modelling has provided one important framework for this purpose. Other fields with the
potential to influence the systematic development of SDC methodology in the future
include data mining, in particular methods related to record linkage, and approaches to
privacy protection in computer science and database technology.

%===========================================%

References
Abowd, J .M. and Lane, J. (2004) New approaches to confidentiality protection: synthetic
data, remote access and research data centers. In J. Domingo-Ferrer and V. Torra
(eds.) Privacy in Statistical Databases. Lecture Notes in Computer Science 3050
Berlin: Springer, pp. 290-297.
Bethlehem, J .G., Keller, W.J. and Pannekoek, J. (1990) Disclosure control for microdata.
J. Amer. Statist. Ass. 85, 38-45.
Blien, U., Wirth, H. and Muller, M. (1992) Disclosure risk for microdata stemming from
official statistics. Statist. Neerland., 46, 69-82.
Bunge, J. and Fitzpatrick, M. (1993) Estimating the number of species: a review. J. Amer.
Statist. Ass. 88, 364-373.
Chen, G, and Keller-McNulty, S. (1998) Estimation of identification disclosure risk in
microdata. J. Off Statist. 14, 79-95.
Cox, L.H. (2001) Disclosure risk for tabular economic data. In P.Doyle, J.I. Lane, J.J.M.
Theeuwes and L.V. Zayatz (eds.) Confidentiality, Disclosure and Data Access:
Theory and Practical Applications for Statistical Agencies. Amsterdam: Elsevier, pp.
167-183.
Cox, L.H., Kelly, J.P. and Patil, R. (2004) Balancing quality and confidentiality for
multivariate tabular data. In J. Domingo-Ferrer and V. Torra (eds.) Privacy in
Statistical Databases. Lecture Notes in Computer Science 3050 Berlin: Springer, pp.
87-98.
De Waal, A.G. and Willenborg, L.C.R..l. (1997) Statistical disclosure control and
sampling weights. J. Ofi’. Statist. 13, 417-434.
Dalenius, T. and Reiss, S.P. (1982) Data-swapping: a technique for disclosure control. J.
Statist. Plan. Inf 6, 73-85.
Defays, D. and Anwar, M.N. (1998) Masking microdata using micro-aggregation. J. Ofi‘.
Statist. 14, 449-461.
Dobra A., Karr A.F. and Sanil A.P. (2003) Preserving confidentiality of high-dimensional
tabulated data: statistical and computational issues. Statistics and Computing, 13, 363-
370.
Dohrmann, S., Curtin, L.R., Mohadjer, L., Montaquila, J . and Le, T. (2002) National
Health and Nutrition Examination Survey: limiting the risk of data disclosure using
replication techniques in variance estimation. Proc. Surv. Res. Meth. Sect. Am. Statist.
Ass., 807-812.
Duncan, G.T. and Lambert, D. (1986) Disclosure-limited data dissemination. J. Amer.
Statist. Ass. 81, 10-28.
Duncan, G.T. and Lambert, D. (1989) The risk of disclosure for microdata. J. Bus. Econ.
Statist. 7, 207-217.
Duncan, G.T., Fienberg, S.E., Krishnan, R., Padman, R. and Roehrig, S.F. (2001)
Disclosure limitation methods and information loss for tabular data. In P.Doyle, J .I.
Lane, J .1 .M. Theeuwes and L.V. Zayatz (eds.) Confidentiality, Disclosure and Data
Access: Theory and Practical Applications for Statistical Agencies. Amsterdam:
Elsevier, pp. 135-166.
Elamir, E.A.H. and Skinner, C.J. (2006) Record level measures of disclosure risk for
survey microdata. J. Off Statist. 22, 525-539.
Evans, T., Zayatz, L. and Slanta, J. (1998) Using noise for disclosure limitation of
establishment tabular data. J. Ofi‘. Statist. 14, 537-551.
Federal Committee on Statistical Methodology (2005) Report on Statistical Disclosure
Limitation Methodology (Statistical Policy Working Paper 22, 2"“ Version)
Washington, D.C.: U.S. Office for Management and Budget.
Fienberg, S.E. and Makov, U.E. (1998) Confidentiality, uniqueness and disclosure
limitation for categorical data. J. Off Statist. 14, 385-397.
Fuller, W.A. (1993) Masking procedures for microdata disclosure limitation. J. Off.
Statist. 9, 383-406.
Giessing, S. (2001) Nonperturbative disclosure control methods for tabular data. In
P.D0yle, J.I. Lane, J.J.M. Theeuwes and L.V. Zayatz (eds.) Confidentiality,
Disclosure and Data Access: Theory and Practical Applications for Statistical
Agencies. Amsterdam: Elsevier, pp. 185-213.
Giessing, S. and Dittrich, S. (2006) Harmonizing table protection: results of a study. In J.
Domingo-Ferrer and L. Franconi (eds.) Privacy in Statistical Databases. Lecture
Notes in Computer Science 4302 Berlin: Springer, pp. 35-47.
Goodman, L.A. (1949) On the estimation of the number of classes in a population. Ann.
Math. Statist., 20, 572-579.
GouWeleeuW,J.M., Kooiman,P., Wi1lenborg,L.C.R.J. and de Wolf, P.-P. (1998) Post
randomisation for statistical disclosure control: theory and implementation, J. Ofi‘.
Statist. 14, 463-478.
Greenberg and Zayatz (1992) Strategies for measuring risk in public use microdata files.
Statist. Neerland., 46, 33-48.
Lambert, D. (1993) Measures of disclosure risk and harm. J. Ofi’. Statist. 9, 313-331.
Lu, W.W., Brick, M. and Sitter, R.R. (2006) Algorithms for constructing combined strata
variance estimators. J. Amer. Statist. Ass. 101, 1680-1692.
Marsh, C., Dale, A. and Skinner, C.J. (1994) Safe data versus safe setting: access to
microdata from the British Census. Int. Statist. Rev. 62, 35-53.
Massell, P., Zayatz, L. and Funk, J. (2006) Protecting the confidentiality of survey tabular
data by adding noise to the underlying microdata: application to the Commodity Flow
Survey. In J. Domingo-Ferrer and L. Franconi (eds.) Privacy in Statistical Databases.
Lecture Notes in Computer Science 4302 Berlin: Springer, pp. 304-317.
Mitra,R. and Reiter,J.P. (2006) Adjusting survey weights when altering identifying
design variables via synthetic data. In J. Domingo-Ferrer and L. Franconi (eds.)
Privacy in Statistical Databases. Lecture Notes in Computer Science 4302 Berlin;
Springer, pp. 177-188.
National Research Council (2005) Expanding Access to Research Data: Reconciling
Risks and Opportunities. Panel on Data Access for Research Purposes, Committee on
National Statistics. Washington, DC: The National Academies Press.
National Statistics (2004), Code of Practice: Protocol on Data Access and
Confidentiality, Norwich, United Kingdom: Her Majesty’s Stationary Office.
Paass, G. (1988) Disclosure risk and disclosure avoidance for microdata, J. Bus. Econ.
Statist. 6, 487-500.
Raghunathan, T.E., Reiter, J. P., and Rubin, D.R. (2003) Multiple imputation for
statistical disclosure limitation, J. Off Statist. 19, 1-16.
Reiter, J .P. (2002) Satisfying disclosure restrictions with synthetic data sets, J. Ofi‘.
Statist. 18, 531-543.
Reiter, J .P. (2003) Inference for partially synthetic, public use microdata sets. Survey
Methodology, 29, 181-188.
Reiter, J . P. (2005) Estimating risks of identification disclosure in microdata, J. Amer.
Statist. Assoc., 100, 1 103-1 1 12.
Salazar, J .1 . (2003) Partial cell suppression: a new methodology for statistical disclosure
control. Statistics and Computing, 13, 13-21.
Samuels, S.M. (1998) A Bayesian species-sampling-inspired approach to the uniques
problem in microdata disclosure risk assessment. J. Ofll Statist., 14, 373-383.
Shlomo, N. and de Waal, T. (2006) Protection of Micro-data Subject to Edit Constraints
Against Statistical Disclosure. Southampton, UK, Southampton Statistical Sciences
Research Institute, 36pp. (S3RI Methodology Working Papers, M06/ 16)
Skinner, C.J. (1992) On identification disclosure and prediction disclosure for microdata.
Statist. Neerland. 46, 21-32.
Skinner, C.]., Marsh, C., Openshaw. S. and Wymer, C. (1994) Disclosure control for
census microdata. J. OflI Statist. 10, 31-51.
Skinner, C.J. (2007) The probability of identification: applying ideas from forensic
science to disclosure risk assessment. J. Roy. Statist. Soc, Ser. A, 170, 195-212.
Skinner, C.J. and Holmes, D.J. (1993) Modelling population uniqueness. International
Seminar on Statistical Confidentiality, Proceedings, European Community Statistical
Office, Luxembourg, 175-199.
Skinner, C.J. and Holmes, D.J. (1998) Estimating the re-identification risk per record in
microdata. J. Ofi‘. Statist. 14, 361-372.
Skinner, C.J. and Carter, R.G. (2003) Estimation of a measure of disclosure risk for
survey microdata under unequal probability sampling. Survey Methodology, 29, 177-
180.
Skinner, C.]. and Elliot, M.J. (2002) A measure of disclosure risk for microdata, J. Roy.
Statist. Soc., Ser. B, 64, 855-867.
Skinner, C.J. and Shlomo, N. (2008) Assessing identification risk in survey microdata
using log-linear models. Journal of American Statistical Association, 103, in press.
Skinner, CJ. and Shlomo, N. (2007) Assessing the disclosure protection provided by
misclassification and record swapping. Paper presented at 56m Session of
Intemational Statistical Institute, Lisbon; to appear in Bulletin of International
Statistical Institute.
Sullivan, G.R. and Fuller, W.A. (1989) The use of measurement error to avoid disclosure.
Proc. Surv. Res. Meth. Sect. Am. Statist. Ass. 802-807.
Van den Hout, A., and Van der Heijden, P. G. M. (2002) Randomized response,
statistical disclosure control, and misclassification: a review. International Statistical
Review, 70, 269-288.
Willenborg, L. and de Waal, T. (2001) Elements of Statistical Disclosure Control. New
York: Springer.
19



Winkler, W.E. (2004) Re-identification methods for masked microdata. In J. Domingo-
Ferrer and V. Torra (eds.) Privacy in Statistical Databases. Lecture Notes in
Computer Science 3050 Berlin: Springer, pp. 216-230.
Yung, W. (1997) Variance estimation for public use files under confidentiality
constraints. Proe. Surv. Res. Meth. Sect. Am. Statist. Ass., 434-439.
Zayatz (1991) Estimation of the number of unique population elements using a sample
Proc. Surv. Res. Meth. Sect. Am. Statist. Ass. 369-373.
20



