The module is expected to include sessions addressing the following topics:

\begin{enumerate}[(i)]
\item  Introduction to basic statistics for clinical trials\item  Types of data summary and data presentation
\item  Probability: Evaluating the role of chance\item  The normal or Gaussian distribution\item  The binomial distribution (optional)
\item  Principles of statistical inference. Point and interval estimation\item  Inference from a sample mean\item  Comparison of two means\item  Comparison of two proportions\item  Association between two categorical variables\item  Measures of effect in 2x2 tables\item  Correlation and linear regression\item  Introduction to survival analysis\item  Allowance for baseline values

\end{enumerate}
%--------------------------------------------
The module will define probability and describe examples of its use. The normal
distribution (and optionally, the binomial distribution) and their application will
be explored, and the principles of statistical inference, including point and
interval estimation, and the role of sampling variation, will be explained. As part
of this introduction, a student will have the option to carry out basic data
analyses from clinical trials using the Stata software package.
