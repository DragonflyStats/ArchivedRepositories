Trial Designs
%% -http://www.lshtm.ac.uk/study/distancelearning/modules/dlmodules/ctm_202_2015.pdf

Keywords Clinical Trial Designs, Early Phase Trials, Late Phase Trials, Parallel Group
Designs, Multi-Arm Designs, Dose Ranging Designs, Cross-Over Designs,
Factorial Designs, Cluster Randomized Controlled Trials, Non-inferiority and
Equivalence Designs, Adaptive Trial Designs.

\subsection{AIMS, OBJECTIVES AND AUDIENCE}
Overall aim An increasing variety of designs is used in clinical trials, and this module will
enable students to understand their fundamental characteristics and
appropriate use in the testing of therapies and other interventions.

Intended learning
outcomes
By the end of this module, students should be able to:
 appreciate the use of different trial designs in assessing interventions and
therapies,
 demonstrate a thorough understanding of the different designs used,
including the strengths and weaknesses of each design,
 appreciate that the sample size requirements depend on the trial design
employed,
 understand that the statistical methods used to analyse a trial rely on the
design used,
 apply appropriate analysis techniques for the analysis of data from trials
using different designs,
 interpret and report results from the analysis of each design of trial.
