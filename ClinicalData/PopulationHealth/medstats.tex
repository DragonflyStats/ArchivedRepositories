\section*{Medical Statistics and Biometry (MSB)}
\subsection*{Clinical trials}

\begin{itemize}
\item Phase I to phase IV clinical trials.
\item Design of phase III clinical trials. Use of controls and blinding. Importance of randomisation.
\item Methods of randomisation: stratification, minimisation, permuted block.
\item Trial protocol and clinical record form (CRF); items to be included in a protocol.
\item Design and analysis of cross-over trials; carry over effects.
\end{itemize}
%==================================%
 
Description of scope of phase I to phase IV trials.
Two period, two treatment trials. t-tests for continuous responses. Treatment × period interaction
and carry over. Advantages and disadvantages of cross-over designs relative to
parallel group studies.
A priori plausible treatment effect and trial size
calculation for qualitative and quantitative data.
 
Sample size calculations for comparing two means or two proportions and for achieving
required precision in estimating a mean or a proportion.
 
Multiplicity: multiple treatments, multiple endpoints, interim analyses.
Data Safety & Monitoring Board and early stopping of clinical trial: surrogate endpoints
Statistical overview (meta-analyses).
 
General ideas on combining results from different studies. Methodological details not required.
 
\subsection*{Epidemiology}
Incidence and prevalence.Design and analysis of cohort (prospective) studies.Design and analysis of case-control (retrospective) studies. Matched case control design and analyses, using McNemar's test.
Causation.
Relative risk and odds ratio.
Confounding and interaction.
\subsubsection*{Mantel-Haenszel procedure}
 
Distinction between these concepts.
Use of logistic regression not required.
Estimation of odds ratio from 2 × 2 tables.
Combining 2 × 2 tables.
Reasons for matching; advantages and
disadvantages relative to unmatched studies.
Inferring causality from observational studies.
Estimation of and confidence intervals for odds ratios.
Use of procedure in adjusting for confounding variables.
%===================================================%
\subsubsection*{Diagnostic tests}
Sensitivity, specificity, ROC curves, positive predictive value.
Uses in diagnosis and in screening.
 
%===================================================%
\subsection*{Analysis of survival data}
Hazard and survivor functions.
Kaplan-Meier estimate of survivor function.
Confidence intervals for survivor function using Greenwood's formula.
 
Logrank test. Parametric survival distributions - exponential, Weibull.
Use of log cumulative hazard plot to check Weibull and proportional hazards assumptions.
Use of these distributions in modelling survival data. Fitting methods not required.
 
Proportional hazards and Cox regression. An understanding of the assumptions and
interpretation of the fitted model. Details of partial likelihood and numerical methods for
fitting the model not required. Calculation and interpretation of hazard ratios and confidence
intervals
.
Checking for non-proportionality of hazards.
Checking for non-proportional hazards using a log cumulative hazard plot and plots of hazard functions.
 
\subsection*{Health information}
International Classification of Diseases.Sources and limitations of data on mortality and
morbidity e.g. cancer registration. Use of health data in the provision of care. Standardised rates.
 
Direct and indirect standardisation of mortalityand morbidity rates. Confidence intervals for
standardised mortality ratio.
 
\subsection*{Option in Biometry}
The aim of this option is to cover the topics of experimental design, survey methods, regression
modelling and bioassay that feature prominently in biometric applications of statistics.
Topics in the syllabuses of the compulsory papers of the Graduate Diploma may be examined with
special reference to biological and agricultural applications.
\subsubsection*{Experimental design}
Further design and analysis of experiments:
principles of design, treatment comparisons and
interpretation of results.
A thorough knowledge of the use of completely
randomised designs, randomised block designs,
Latin squares and factorial experiments, and the
corresponding analysis of variance tables.
Methods of dealing with several missing values;
unbalanced designs.
Analysis of non-orthogonal data; adjusted
treatment effects.
Covariance analysis. Incorporating information on a single covariate.
Relationship of analysis of covariance to the
comparison of regression lines.
.
Nested designs and components of variance. Split
plot experiments.
Estimation of components of variance by equating
observed to expected mean squares.
%===================================================%
\subsection*{Survey methods}
Censuses and sample surveys: planning and
design. Use of maps and aerial surveys; crop
estimation and forecasting; forestry and land use
surveys. Handling large data sets on a computer.
Knowledge of any particular database
management package is not required.
%===================================================%
\subsection*{Regression modelling}
Multiple regression analysis. Comparison of regressions. Use of indicator variables. Weighted
regression.
Use of residual sum of squares in comparing alternative models. Comparison of straight lines.
Variable selection methods. Model checking using residuals. Multicollinearity. Interpretation
of output from packages.
 
Non-linear modelling. Fitting standard growth curves (including logistic and Gompertz).
Estimation of parameters in non-linear models.
An appreciation of the Newton-Raphson scheme for fitting non-linear models.
 
\subsection*{Bioassay}
Biological assay. Continuous and binomial responses; probit, logistic and angular transformations;
median effective dose (ED50), relative potency, parallel line and slope ratio assays, confidence limits for ED50. Analysis of bioassay data using generalised linear models.
 
Greatest emphasis will be placed on the use of logistic regression for analysing binomial data.
Estimation of ED50 from fitted dose response curves. Use of Fieller's theorem, including
derivation of the result. Use of general linear model for continuous data and logistic regression for binary data
 
 %==================================================%
\subsection*{Clinical Trial Protocol}
 
A Clinical Trial Protocol is a document that describes the objective(s), design, methodology, statistical considerations, and organization of a clinical trial. The protocol usually also gives the background and reason the trial is being conducted, but these could be provided in other documents referenced in the protocol (such as an Investigator's Brochure).
The protocol contains a study plan on which the clinical trial is based. The plan is designed to safeguard the health of the participants (while limiting their financial liability) as well as answer specific research questions. The protocol describes, among other things, what types of people may participate in the trial; the schedule of tests, procedures, medications, and dosages; and the length of the study. While in a clinical trial, study participants are seen regularly by the research staff (usually medical doctors and/or nurses) to monitor their health and to determine the safety and effectiveness of the treatment(s) they are receiving.
 
