Basic Statistics for Clinical Trials
CTM102

Introduction to basic statistics for clinical trials
Types of data summary and data presentation
Probability: Evaluation the role of change
The normal or Gaussian distribution
The binomial distribution
Principles of statistical inference.
Point and interval estimation
Inference from a sample mean
Comparison of two means
Comparison of two proportions
Association between two categorical variables
Measures of effect in 2x2 tables
Correlation and linear regression
Introduction to survival analysis
Allowance for baseline values
The module will define probability and describe examples of its use.
Normal and binomial distributions and their application will be explored, and the principles of statistical inference, including point and interval estimation, and the role of sampling variation, will be explained.
As part of this introduction, a student will have the option to carry out basic data analyses from clinical trials using the computer-based Stata software package.

Ethical Considerations
Clinical Trial Designs
Bias and Random Error
Objectives and Endpoints
Sample Size and Power
The Study Cohort
Treatment Allocation and Randomization
Interim Analyses and Stopping Rules
Missing Data and Intent-to-Treat
Estimating Clinical Events
Prognostic Factors and Regression Models
Reporting and Publishing
Factorial Designs
Cross-Over Designs
Overviews and Meta-Analysis
Diagnostic Testing
Measures of Agreement
