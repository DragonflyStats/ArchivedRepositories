Good clinical practice (GCP) is an international quality standard that is provided by ICH, an international body that defines standards, which governments can transpose into regulations for clinical trials involving human subjects. A similar guideline for clinical trials of medical devices is the international standard ISO 14155, that is valid in the European Union as a harmonized standard. These standards for clinical trials are sometimes referred to as ICH-GCP or ISO-GCP to differentiate between the two and the lowest grade of recommendation in clinical guidelines.[1]

GCP follows the International Conference on Harmonisation of Technical Requirements for Registration of Pharmaceuticals for Human Use (ICH) of GCP guidelines. GCP enforces tight guidelines on ethical aspects of a clinical study. High standards are required in terms of comprehensive documentation for the clinical protocol, record keeping, training, and facilities, including computers and software. Quality assurance and inspections ensure that these standards are achieved. GCP aims to ensure that the studies are scientifically authentic and that the clinical properties of the investigational product are properly documented.

GCP guidelines include protection of human rights for the subjects and volunteers in a clinical trial. It also provides assurance of the safety and efficacy of the newly developed compounds.

GCP guidelines include standards on how clinical trials should be conducted, define the roles and responsibilities of clinical trial sponsors, clinical research investigators, and monitors. In the pharmaceutical industry monitors are often called Clinical Research Associates.
